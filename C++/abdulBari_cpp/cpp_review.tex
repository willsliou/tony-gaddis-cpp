\documentclass{article}
\usepackage{amsmath}
\usepackage{amsfonts}
\usepackage{amssymb}
\usepackage{booktabs}
\usepackage[normalem]{ulem}
\usepackage{hyperref} % hyperlinks
\usepackage{longtable}
\hypersetup{
    colorlinks=true,
    linkcolor=blue,
    filecolor=magenta,      
    urlcolor=cyan,
}

\urlstyle{same}

\useunder{\uline}{\ul}{} 
`'
\usepackage{tikz} % for array DS
\usetikzlibrary{calc} % for array DS

\title{C++ Basics (Abdul Bari)}
\author{Wills Liou}
\date{today}

\maketitle
\begin{abstract}
	Summary of C++ Basics by Abdul Bari (Udemy); This took a really long time.
\end{abstract}


\begin{document}
 cout << "Hello World" << endl;	 
	\#include <iostream>
	iostream is a library
	cout stands for "console out"
	COUT and CIN are OBJECTS
		
	
	<< is an insertion operator
	:: is a scope resolution
	
	= is an "ASSIGNMENT OPERATOR"
	== is a "comparison operator"
	
	< is less than
	5 < 3    five is less than 3
	
	> is greater than
	2 > 1
	2 is greater than 1
	
	Program is set of two ingredients: 
		
	1. data
	2. operations performed on data


	% MEMORY	
	1 byte = 8 bits
	
	% BYTES
	Maximum integer in 1 byte (8 bits) is 255.
	
	\href{https://www.sciencedirect.com/topics/engineering/binary-word}{Formula for maximum binary value for N bits} \\
	Formula to calculate maximum decimal value for N bits.
		$$M = 2^{N} - 1$$
	Example:	
		$$255 = 2^{8} - 1$$


\chapter{25. Primitive Data Types}

\begin{table}[]
\begin{tabular}{ccc}
\hline
{Data Type}               & {Size (bytes) }               & {Range}               \\ \hline
\textbf{int}            & \textbf{2 or 4}            & \textbf{-32,768 -to 32,767}      \label{int}   \\ 
\textit{float}                    &4                      &$-3.4 \times 10^{-38}$ to $3.4 \times 10^{38}$     \label{float}             \\
\textit{double}                     &8                      & $-1.7 \times 10^{-308}$ to $1.7 \times 10^{308}$       \label{double}     \\ \hline
\textbf{char}            & \textbf{1}            & \textbf{-128 to 127}            \\
bool                     &undefined                      &true/false                      \\
\multicolumn{1}{l}{} & \multicolumn{1}{l}{} & \multicolumn{1}{l}{} 
\end{tabular}
\end{table}

\textbf{How to memorize range for \ref{float} floats and \ref{double} doubles} \\
Realize that doubles have an extra \textbf{zero} in the power \\
floats are \textbf{$10^{38}$} and doubles are \textbf{$10^{308}$}

\textbf{Float range}: $-3.4 \times 10^{-38}$ to $3.4 \times 10^{38}$ \\
\textbf{Double range}:  $-1.7 \times 10^{-308}$ to $1.7 \times 10^{308}$ \\
\textbf{multiply} double's range and get float's number;  \\
$1.7 \times 2 = 3.4$
Other data types: 1. long 2.long long (for really long decimals)


\section{Proving the int range is {-32,768 -to 32,767} } 
\begin{table}
\centering
\begin{tabular}{lllllllllllllllllll}
 &16  &15  &  &  &  &  &  &  &  &  &  &  &  &  &  &   
\end{tabular}
\end{table}

This is our array for 2 bytes (16 bits) \\
Assume the numbers inside each index are it's index value \\
Notation:
Most significant byte (MSB), Least significant byte (LSB) \\
-ve stands for negative
+ve stands for positive

Known: 1 byte = 8 bits. \\
If we have two bytes, we will have 16 bits. Let's assume that our int data type takes in two bytes (in some old compiler).

\newcounter{nodeidx}
\setcounter{nodeidx}{1}
\newcommand{\nodes}[1]{%
    \foreach \num in {#1}{
      \node[minimum size=6mm, draw, rectangle] (\arabic{nodeidx}) at (\arabic{nodeidx},0) {\num};
      \stepcounter{nodeidx}
    }
}
\newcommand{\brckt}[4]{% from, to, lvl, text
  \draw (#1.south west) ++($(-.1, -.1) + (-.1*#3, 0)$) -- ++($(0,-.1) + (0,-#3*1.25em)$) -- node [below] {#4} ($(#2.south east) + (.1,-.1) + (.1*#3, 0) + (0,-.1) + (0,-#3*1.25em)$) -- ++($(0,#3*1.25em) + (0,.1)$);%
}
  \begin{tikzpicture}
    \pgftransparencygroup
    \nodes{15,14,13,12,11,10,9,8,7,6,5,4,3,2,1,0}
    \endpgftransparencygroup
    \pgfsetstrokeopacity{0.5}
    \pgfsetfillopacity{0.5}
    \pgftransparencygroup
    \endpgftransparencygroup
    \pgfsetstrokeopacity{.75}
    \pgfsetfillopacity{.75}
    \pgftransparencygroup
    \brckt{1}{1}{0}{MSB (+) or (-)}
    \endpgftransparencygroup
    \pgfsetfillopacity{0.5}
    \pgfsetstrokeopacity{0}
    \pgftransparencygroup
	\brckt{2}{16}{0}{15 bits left}
	 \brckt{16}{16}{0}{LSB}
    \endpgftransparencygroup
    \pgfsetstrokeopacity{0.5}
    \pgftransparencygroup
    \brckt{1}{16}{1}{16 bits total}
    \endpgftransparencygroup
  \end{tikzpicture}

  \newcommand{\lep}{\hspace*{.5em}}
  \noindent
  $\fbox{5} \lep \fbox{2} \lep \fbox{7} \lep \fbox{-5} \lep \fbox{16} \lep \fbox{12}$

\begin{enumerate}
	\item Assume 2 bytes (16 bits)
	\item Most significant bit (MSB) reserves the first spot for the sign; In other words, MSB takes one spot for a (1) -ve or (0) +ve
	\item Excluding the MSB, there are 15 bits left for the integer.
	\item Therefore $2^{15} = 32768$	
		\begin{enumerate}
			\item $32,768$ is the maximum number from the \textit{range 1 to 32,768}
			\item So, our int data type can take in a max number, 32,768.
		\end{enumerate}
	\item If we want a range starting from 0, our range is now \textit{range of 0 to 32,767}
	\item   Range visually represented as:  0     \hspace{1cm}  ...     32,767
	\item Reminder: we know that one bit is reserved at MSB (first index) for positive and negative signs. 
	\item  So if we include the negative side from -32767 to -0, our range looks like this
	\item -32,767  \hspace{1cm}        -0 \hspace {1cm} 0 \hspace {1cm} 32,767
	
	\item But realize that -0 does not exist; we add one more value to the negative side to account for our -0
	\item -32,767 + 1  \hspace {1cm} 0 \hspace {1cm} 32,767	
	\item -32,768  \hspace {1cm} 0 \hspace {1cm} 32,767
	\item Therefore our int data type range is from \textit{-32,768 to 0 to 32,767}
\end{enumerate}	

Question at step \#9: How do we have enough memory for the negative side if our maximum number is 32,768? \\
Answer: Look at two's complement; when we reach our one more than maximum value of 32,768, (e.g. 32,769), we actually overflow and get -1. 

\href{https://www.cs.cornell.edu/~tomf/notes/cps104/twoscomp.html}{Two's Complement} \\

In practice, it looks like this:
\begin{center}
\begin{table}
    \begin{tabular}{|c|c|c|c|c|c|c|c|c|c|c|c|c|c|c|c|} % 'c' stands for center; 'l' stands for left align
        \hline
        ~1 & ~0 & ~0 & ~0 & ~0 & ~0 & ~0 & ~0 & ~0 & ~0 & ~0 & ~0 & ~0 & ~0 & ~0 & ~1 \\
        \hline
    \end{tabular}
\end{table}
\end{center}

% copy pasta from https://tex.stackexchange.com/questions/79369/formatting-table-border-and-text-alignment-in-latex-table
\newdimen\boxitspace\boxitspace=3pt% boxit from eplain
\long\def\boxit#1{\vbox{\hrule\hbox{\vrule\kern\boxitspace\vbox{%
  \kern\boxitspace\parindent0pt#1\kern\boxitspace}%
  \kern\boxitspace\vrule}\hrule}}
\def\aryitem#1{\boxit{\hbox to 1.4em{\hfil#1\hfil}}}% change 1.4em if needed
\let\MS\multispan% shorten a little
\halign{&\thinspace\aryitem{$\mathstrut#$}\thinspace\cr
  5&2&7&-5&16&12&\omit\quad&?&?&?&?&?&?\cr
  \MS6\upbracefill&\omit&\MS6\upbracefill\cr
  \MS6\hfill size $=6$\hfill&\omit&\MS6\hfill free space\hfill\cr
  \MS{13}\upbracefill\cr
  \MS{13}\hfill capacity $=12$\hfill\cr
}
\bye


\section{Proving the char range is -128 to 127}
  \begin{enumerate}
    \item The data type, char, is 1 byte or 8 bits.
    \item We reserve the first bit (MSB) for the positive and negative signs.
    \item Therefore, we are left with 7 bits.
    \item So we have $2^{7} = 128$
    \begin{enumerate}
      \item our values for char is the range \textit{1 to 128}
    \end{enumerate}
    \item When we begin at 0, our range becomes \textit{0 to 127}
    \item To account for the negative side, we have \textit{-127  -0  0  127}
    \item To fix the -0, we add one more value to our negative side, our final range is \textit{-127  0  127}
  \end{enumerate} 
   This had many similiar to our int range proof.

  For unsigned(only positive) integers, add back the first bit that stores the negative and positive values. Aka. use all 16 bits (2 bytes in integer data type)
  One way:
  \begin{enumerate}
  \item Instead of $2^{15} = 32,768$, we will have $2^{16} = 65,536$.
  \item With a range starting from 0, \textit{0 to 65,535}
  \item Therefore, unsigned int has a range from \textit{0 to 65,535}
  \end{enumerate}
  Other way: Add the min and max values of an int range \textit{-32,768 to 0 to 32,767}
  \begin{enumerate}
    \item $32spo
    q ,768 + 32,767 = 65,535$
    \item Therefore, unsigned int has a range from \textit{0 to 65,535}
  \end{enumerate}

  \section{*** IMPORTANT REMEMBER THESE: ***}
   a = 97 ... z = 122; 0 -> 49 ... 9 -> 57
  \begin{center}
    \begin{tabular}{ c c c }
    A = 65 & a = 97 & '0' = 48 \\
    B = 66 & b = 98 & '1' = 49 \\
      .                     \\  
      . \\
      .\\ 
    Z = 90
    \end{tabular}
  \end{center}



  \begin{center}
    Data Types
  \begin{tabular}{ c c }
    Data Type & Range value
    int & -37,768 to 37,767 (15 bits used with 1 reserved) \\
    char & -128 to 127 (8 bits used with 1 reserved)
    unsigned int & 0 to 65,535 (all 16 bits or 2 bytes are used)
    unsigned char & 0 to 255 (all 8 bits are used)
    long int & if int is 2 bytes, then long int is 4 bytes or (if int is 4 bytes, then long int is 8 bytes
    long float (does not exist, it is a double) 
    long double  10 bytes

  \end{tabular}
\end{center}





\end{document}